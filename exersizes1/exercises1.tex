\documentclass{tufte-book}

\usepackage{amsmath, amsthm}
\usepackage{graphicx}
\setkeys{Gin}{width=\linewidth,totalheight=\textheight,keepaspectratio}
\graphicspath{{graphics/}}

\title{Math Camp 2016\\Exercises }
\author{Joe Seidel}
\date{\today}

\usepackage{booktabs}
\usepackage{units}
\usepackage{fancyvrb}
\fvset{fontsize=\normalsize}
\usepackage{multicol}
\usepackage{lipsum}
\usepackage{pdfpages}
\usepackage{tikz}
\usepackage{wasysym}
\usepackage{amssymb}

\newcommand{\doccmd}[1]{\texttt{\textbackslash#1}}% command name -- adds backslash automatically
\newcommand{\docopt}[1]{\ensuremath{\langle}\textrm{\textit{#1}}\ensuremath{\rangle}}% optional command argument
\newcommand{\docarg}[1]{\textrm{\textit{#1}}}% (required) command argument
\newenvironment{docspec}{\begin{quote}\noindent}{\end{quote}}% command specification environment
\newcommand{\docenv}[1]{\textsf{#1}}% environment name
\newcommand{\docpkg}[1]{\texttt{#1}}% package name
\newcommand{\doccls}[1]{\texttt{#1}}% document class name
\newcommand{\docclsopt}[1]{\texttt{#1}}% document class option name
\DeclareMathOperator{\proj}{proj}
\newcommand{\vct}{\mathbf}
\newcommand{\dprod}[2]{\langle \vct{#1}, \vct{#2} \rangle}

\newtheoremstyle{mytheoremstyle} % name
	{\topsep}		% Space above
	{\topsep}		% Space below
	{\itshape}		% Body font
	{}			% Indent amount
	{\bfseries}	% Theorem head font
	{\textnormal{:}}	% Punctuation after theorem head
	{.5em}		% Space after theorem head
	{}			%Theorem headspec
\theoremstyle{mytheoremstyle}
\newtheorem*{thm}{Thm.}

\newtheoremstyle{mylemstyle} % name
	{\topsep}		% Space above
	{\topsep}		% Space below
	{\itshape}		% Body font
	{}			% Indent amount
	{\bfseries}	% Theorem head font
	{\textnormal{:}}	% Punctuation after theorem head
	{.5em}		% Space after theorem head
	{}			%Theorem headspec
\theoremstyle{mylemstyle}
\newtheorem*{lem}{Lem.}


\newtheoremstyle{mydefstyle} % name
	{\topsep}		% Space above
	{\topsep}		% Space below
	{\normalfont}	% Body font
	{}			% Indent amount
	{\bfseries}	% Theorem head font
	{\textnormal{:}}	% Punctuation after theorem head
	{.5em}		% Space after theorem head
	{}			%Theorem headspec
\theoremstyle{mydefstyle}
\newtheorem*{mydef}{Def.}
\newtheorem*{ex}{E.g.}

\begin{document}

\maketitle
\pagenumbering{gobble}
\newpage
\pagenumbering{arabic}

\subsection{Exercise 1.1}
Prove that
\[ \| \vct{x} +\vct{y} \|^2 + \|\vct{x}-\vct{y}\|^2 = 2\|\vct{x}\|^2 + 2\|\vct{y}\|^2 \]
if $\vct{x} \in R^k$ and $\vct{y} \in R^k$.

\begin{proof}\marginnote{\it{norm}: $\|x\| = (\dprod{x}{x})^{\frac{1}{2}}$}
\begin{align*}
\| \vct{x} +\vct{y} \|^2 + \|\vct{x}-\vct{y}\|^2 &=  \dprod{x+y}{x+y} + \dprod{x-y}{x-y} \\
&= \sum_1^k(x_i+y_i)^2 + \sum_1^k(x_i - y_i)^2\\
&= \sum_1^k x_i^2 + \sum_1^k 2x_iy_i + \sum_1^k y_i^2 + \sum_1^k x_i^2 - \sum_1^k 2x_iy_i + \sum_1^k y_i^2 \\
&=  2\sum_1^k x_i^2 + 2\sum_1^k y_i^2\\
&= 2\dprod{x}{x} + 2\dprod{y}{y}\\
&= 2\|\vct{x}\|^2 + 2\|\vct{y}\|^2\\
\end{align*}

Consider parrallegram formed by $\vct{x}$ and $\vct{y}$ in $\mathbb{R^2}$. The sum of the length of the sides, squared will be equal to the sum of the diagonals squared.

\end{proof}

\subsection{Exercise 1.2}
For $x \in \mathbf{R}^1$ and $y \in mathbf{R}^1$, determine whether each is a metric.

\begin{enumerate}
\item $d_1(x,y) = (x-y)^2$ \\
Fails subadditivity.  Consider when
$x=4$, $y=-1$, $z=2$.
\[ d(4,-1) = 25 > 15 = d(4,2) + d(2,-1) \]

\item $d_2(x,y) = \sqrt{|x-y|}$
This fails non-negativity.  Suppose, $x=12$ and $y = 3$.

\[ d(12,3) = \sqrt{|12-3|} = \pm 3 \]

For subadditivity, first observe

\begin{align}
|x - y| &= |x - y + z - z|\\
&= |x - z + z - x| \\
&\leq |x - z| + |z - x|\\
&\leq |x - z| + |z - x| + 2\sqrt{|x-z|}\sqrt{|z-y|}\\
&= (\sqrt{|x-z|} + \sqrt{|z-y|})^2
\end{align}

Taking the square root of the resulting inequality
\[ \sqrt{|x-y|} \leq \sqrt{|x-z|} + \sqrt{|z-y|} \]

Line 4, is not neccessarily true.

\item $d_3(x,y) = |x^2 - y^2|$

It is obvious this definition satisifies nonnegativity and symmetry.

For subbadditivity, observe
\begin{align*}
|x^2 - y^2| &= |x^2 - z^2 + z^2 - y^2| \\
&\leq |x^2 - z^2| + |z^2 - y^2|
\end{align*}

Hence $d(x,y) \leq d(x,z) + d(z,y)$

\end{enumerate}

\subsection{Exercise 3}
Consider the following subset of $\mathbb{R}^2$ and discuss the closedness, openes, and boundedness of each.

\begin{enumerate}

\item $E = \mathbb{N}$ The set of all integers.

Closedness:  from (b), the set has no limit points.  Hence, every limit point is contained in E.  Therefore it it is closed.

Openness: E has no interior points, hence E is not open. $\exists x \in E$ such that $x$ is not an interior point.

Boundedess:  For $\forall x \in N$ there exists $y$ such that $x<y$.  (Archmedian principle).  Unbounded.

\item $E = \{\frac{1}{n} \ | \ n \in \mathbb{N}^+ \}$

\textit{Closedness}: $0$ is an accumulation point of $E$, but $0 \notin E$ therefore $E$ is not closed.

\textit{Openness}: $\forall x \in E$, $\exists r > 0$ such that $N_r(x) \not\subset E$ by the properaty for $\mathbb{Q}$ is dense in $\mathbb{R}$.  Hence $E$ has no interior points and is not open.

\textit{boundedness}:   $E$ is bounded.  Choose $M=2$, observe $\forall x \in E$, $d(x, (0,0)) < M$.

\item $E= \mathbb{R}^2$
This set is open, closed and unbounded.  The first two easily follow from (d) and (f).  For boundedness, observe that given any $M$ such that $d(x, p) < M$ for $x \in E$ there exists a $p \in \mathbb{R}^2$ such that $d(x,p) > M$, using the standard metric.  However, in the discrete metric this would be bounded.
\end{enumerate}

\subsection{Exercise 4}
Determine whether each of the following sets is compact.

\begin{enumerate}

\item $[0,1]$\\
Closed and bounded, hence compact

\item $[0,1)$
Observe that $1$ is a limit point of the set.  The set is not closed, so not compact.  Also, let $B_n = (0, \frac{1}{n})$ for $n \in \mathbb{N}^+$  Observe that the the open cover of the set, $\{0\} \cup \bigcup_n^{\infty} B_n$ has no finite subcover.

\item $E = \{ 1, 2, 3\}$
Compact.   The set is closed and bounded.

\item $E = \{ \frac{1}{n} \ | \ n=1,2,3... \}$
Observer the $0$ is a limit point of $E$, but $0 \notin E$ so $E$ is not closed. Therefore, not compact.

\item $E = \{ \frac{1}{n} \ | \ n=1,2,3... \} \cup \{0\}$  Closed and bounded, therefore compact.

\subsection{Exercise 5}
Calculate $\lim_{n \to \infty}(\sqrt{n^2+n}-n)$

First note
\begin{align*}
\sqrt{n^2+n}-n &= \sqrt{n^2+n}-n \frac{\sqrt{n^2+n}+n}{\sqrt{n^2+n}+n}\\
&=\frac{n}{\sqrt{n^2+n}+n}\\
&= \frac{1}{\sqrt{1 + \frac{1}{n}} + 1}\\
\end{align*}

Next, observe
\[ \lim_{n \to \infty} \frac{1}{n} = 0 \]

Therefore
\[\lim_{n \to \infty}(\sqrt{n^2+n}-n) = \frac{1}{2} \]

\end{enumerate}

\subsection{Exercise 6}
Given
\[ x_n = 1 + \frac{1}{2^2} + ... + \frac{1}{n^2} \]

prove the sequence $\{x_n\}$ converges.

\begin{proof}

Choose any $\epsilon > 0$. Then given sufficiently large $N \in \mathbb{N}$
\[ d(x_n, x_{n+1}) = \frac{1}{(n+1)^2} < \epsilon \]
where $n \geq N$.  Hence the sequence is Cauchy and, equivalently, convergent.

\end{proof}

\subsection{Exercise 7}
If $s_1 = \sqrt{2}$ and $s_{n+1} = \sqrt{2 + s_n}$ for $n=1,2,3...$ prove that $\{s_n\}$ converges, and and that $s_n < 2$ for $n=1,2,3...$.

First, prove monotonicity using induction, i.e. $s_n < s_{n+1}$.
Setting $n=1$
\[ s_1 = \sqrt{2} < \sqrt{2 + \sqrt{2}} = s_2 \]
as required.

Let $n$ be an arbitrary natural number and suppose that $s_n < s_{n+1}$.  Then
\begin{align*}
s_{n+1} = \sqrt{2 + s_n} < \sqrt{2 + s_{n+1}} = s_{n+2}
\end{align*}

Since both the base case and the inductive have been performed, through mathematical induction, $s_n < s_{n+1}$ holds for all natural numbers.

Therefore, $\{s_n\}$ is monotonic.

Next show that $\{s_n\}$ is bounded.  We are given that it is bounded below by $\sqrt{2}$. Using induction, show that $s_n < 2$ for $n=1,2,3,...$.  Setting $n=1$

\[ s_1 = \sqrt{2} < 2 \]

Let $n$ be arbitrary and suppose $s_n < 2$.  Then

\begin{align*}
s_{n+1} &= \sqrt{2 + s_n}\\
&< \sqrt{2 + 2}\\
&=2
\end{align*}

Since both the base case and the inductive have been performed, through mathematical induction, $s_n < 2$ holds for all natural numbers.

Hence, $\{s_n\}$ is bounded above by $2$ and converges.

\subsection{Exercise 8}
Find the upper and lower limits of the sequence $\{s_n\}$ defined by
\[ s_n = \frac{(-1)^n}{1+\frac{1}{n}} \]

Observe that the odd numbered indexed sequences are negative and approach $-1$ while the even number indexed sequences are even and approach $1$. Hence

\[ \limsup s_n = 1 \text{ and } \liminf s_n = -1 \]

\subsection{Exercise 9}
Calculate
\[ \sum_{n=0}^{\infty}(n+1)x^n \text{ for } 0 \leq x < 1 \]

We need to find a general formula of $S_n = \sum_{n=0}^{\infty}(n+1)x^n$.  It will be usefull to recall the gemotric series

\[ \sum_{i=0}^{n} r^i = \frac{1-r^{n+1}}{1-r} \text{ for } (-1 < r < 1) \]
and
\[ \sum_{i=0}^{\infty} r^i = \frac{1}{1-r} \]

Calculating for $S_n$
\begin{align}
\sum_{n=0}^{\infty}(n+1)x^n &= 1 + 2x + 3x^2 + ...\\
&= 1 + x + x^2 ... + x(1 + x + x^2...) + x^2(1 + x + x^3...) +...\\
&= (1 + x + x^2 + ...)^2 \\
&= (\sum_{i=0}^{\infty} x^i)^2\\
&= \frac{1}{(1-x)^2}
\end{align}

It is important to note that step (8) to (9) works because we are given $0 \leq x < 1$.


\subsection{Exercise 10}

Let $x_n = \sum_{k=0}^n \frac{1}{n!}$.  Prove that $\{x_n\}$.  Converges.

Observe that
\[ \frac{1}{n!} \leq \frac{1}{n(n-1)} \text{ for } n = 2, 3,4,5... \]

Which means

\begin{align*}
\sum_{k=2}^n \frac{1}{n!} &= \frac{1}{2!} + ... + \frac{1}{n!}\\
&\leq \frac{1}{2} + ... + \frac{1}{n(n-1)} \\
&= \sum_{k=2}^n \frac{1}{k(k-1)} \\
&= 1 - \frac{1}{n}\\
&< 1
\end{align*}

Since $\sum_{k=0}^1 \frac{1}{n!} = 2$ we can conlude that
\[ \sum_{k=0}^n \frac{1}{n!} < 3 \]

Hence $\{x_n\}$ is bounded and monotonic sequence and therefore converges.

\subsection{Exercise 11}
Determine if the series $\sum a_n$ is convergent or divergent where
\[ a_n = \frac{\sqrt{n+1} - \sqrt{n}}{n} \]
First observe
\begin{align*}
a_n &= \frac{\sqrt{n+1} - \sqrt{n}}{n} \cdot \frac{\sqrt{n+1} + \sqrt{n}}{\sqrt{n+1} + \sqrt{n}}\\
&= \frac{1}{n(\sqrt{n+1} + \sqrt{n})}\\
\end{align*}

It's helpful to know the sequence \marginnote{For proof of this, see Week2 notes from Math 203}
\[ \sum_{n=1}^{\infty} \frac{1}{n^p} \]
converge when $p > 1$.

By the comparison test

\[ \frac{1}{n(\sqrt{n+1} + \sqrt{n})} = < \frac{1}{n^{3/2}} \]

The sequence converges.

\subsection{Exercise 12}

Determine whether the series $\sum a_n$ is convergent or divergent, where

\[ a_n = (\sqrt[n]{n} - 1)^n .\]

Using the ratio test

\begin{align*}
\alpha &= \limsup_{n \to \infty} \sqrt[n]{|a_n|} \\
&= \limsup_{n \to \infty} \sqrt[n]{n} - 1\\
\end{align*}

A useful side note,
\begin{align*}
n^{\frac{1}{\log n}} &= x \\
\frac{1}{\log n} \log n &= \log x\\
1 &= \log x\\
x &= e
\end{align*}

Which we can employ in calculating
\begin{align*}
\alpha &= \limsup_{n \to \infty} \sqrt[n]{n} - 1\\
&= \limsup_{n \to \infty} (e^{\frac{\log n}{n}}) - 1\\
&= \exp\big(\limsup_{n \to \infty} \frac{\log n}{n} \big) - 1 \\
&= 0
\end{align*}

Since $\alpha = 0 < 1$, the root test allows us to conlude that the series converges.

\subsection{Exercise 13}
Determine whether the series $\sum a_n$ converges or diverges, where
\[ a_n = \frac{(-10)^n}{4^{2n+1}(n+1)} \]

Using the ratio test, $L = \lim_{n \to \infty} \big| \frac{a_{n+1}}{a_n} \big|$.

Compute

\[ \frac{a_{n+1}}{a_n} = \frac{(-10)(n+1)}{16(n+2)} \]

\[  \lim_{n \to \infty} \big| \frac{(-10)(n+1)}{16(n+2)} \big| = \frac{10}{16} \]

Since $L=\frac{10}{16} < 1$ the series converges.

\subsection{Exercise 14}
Show that the limit does not exist:
\[ \lim_{x \to 0} \sin\big(\frac{1}{x}\big) \]

Consider the sequences $\{x_n\} = \frac{1}{\pi n}$ and $\{y_n\} = \frac{1}{\frac{\pi}{2}+2\pi n}$.

\[ \lim_{n \to \infty} x_n = 0 \text{ and } \lim_{n \to \infty} y_n = 0 \]

However,

\[ \lim_{x \to 0} \sin\big(\frac{1}{x_n}\big) = 0 \text{ and } \lim_{x \to 0} \sin\big(\frac{1}{y_n}\big) = 1   \]

therefore the limit does not exist.

\subsection{Exercise 15}
Compute
\[ \lim_{x \to 0} \frac{ \sqrt{1+x} - \sqrt{1-x} }{x} \]

Observe
\begin{align*}
\frac{ \sqrt{1+x} - \sqrt{1-x} }{x} &= \frac{ \sqrt{1+x} - \sqrt{1-x} }{x} \cdot \frac{\sqrt{1+x} + \sqrt{1-x} }{\sqrt{1+x} + \sqrt{1-x} }\\
&= \frac{2x}{x(\sqrt{x+1} + \sqrt{1-x}) }\\
&= \frac{2}{\sqrt{x+1} + \sqrt{1-x}}\\
\end{align*}

Hence,
\[ \lim_{x \to 0} \frac{2}{\sqrt{x+1} + \sqrt{1-x}} = 1 \]

\subsection{Exercise 18}
Suppose \marginnote{Exercises 16 and 17 don't appear in the notes I am following} that $f(x)$ is continuous on $[a,b]$.  Let
\[ \eta = \frac{1}{3} [f(x_1) + f(x_2) + f(x_3)] \]

where $x_1, x_2, x_3 \in [a,b]$.  Prove that there exists $c \in [a,b]$ such that $f(c) = \eta$.

Notice that $f$ is a continous mapping of over the closed interval $[a,b]$, then $f$ attains it's min and maximum values.  Let $f(d) = m$ and $f(c) = M$, the min and max values of the functions.  Hence, $f(d) \leq f(x) \leq f(c)$ for all $x \in [a,b]$.

Let $x_1 \neq x_2 \neq x_3$, observe
\[ f(c) = \frac{1}{3}[ f(c) + f(c) + f(c) ] < \frac{1}{3}[ f(x_1) + f(x_2) + f(x_3)] \]
and
\[ f(d) = \frac{1}{3}[ f(d) + f(d) + f(d) ] > \frac{1}{3}[ f(x_1) + f(x_2) + f(x_3)]. \]

Since $f(c) < \eta < f(d)$, by the Intermediate Value Theorem, there exists $c \in [a,b]$ such that $f(c) = \eta$.

\subsection{Exercise 19}
For $x,y \in \mathbb{R}$, define
\[ d'(x,y) = \frac{|x-y|}{1+|x-y|}. \]

Determine whether it is a metric or not.

Non-negativity and symmetry are obvious. For subadditivity, let $d(x,y)=|x-y|$, observe
\begin{align*}
d'(x,z) + d'(z,y) &= \frac{d(x,z)}{1+d(x,z)} + \frac{d(z,y)}{1+d(z,y)}\\
&\geq \frac{d(x,z)}{1+d(x,z) + d(z,y)} + \frac{d(z,y)}{1+d(x,z)+d(z,y)}\\
&= \frac{d(x,z) + d(z,y)}{1+d(x,z) + d(z,y)}\\
&= 1 - \frac{1}{1+d(x,z) + d(z,y)}\\
&\geq 1 - \frac{1}{1+d(x,y)}\\
&= \frac{d(x,y)}{1+d(x,y)}\\
&= d'(x,y)\\
\end{align*}

Therefore, $d'(x,y)$ is a metric.

\subsection{Exercise 20}
Let X be an infinite set.  For $p,q \in X$, define

\[ d(p,q) =
\begin{cases}
1, & \text{ if } p \neq q \\
0, & \text{ if } p=q\\
\end{cases}
\]

Prove that this is a metric. Which subsets of the resulting metric space are open?  Which are closed?  Which are compact?

Non-negativity and symmetry are obvious. For subadditivity prove by contradiction.  Assume
\[ d(x,y) > d(x,z) + d(y,z) \]

If $x=y$, there is an immediate contradiction.  If $x\neq y$ then we must have $x=z$ and $y=z$, but then $x=y$ which is a contradiction.

Therefore

\[ d(x,y) \leq d(x,z) + d(y,z) \]

All sets are open and closed. Let $A \subset X$  Since any ball of $e<1$ around a point, $N_r(x) \subset A$, therefore every subset of $X$ is the singleton $\{x\}$ and is open. Then $A^c = X \setminus A$ is open.  So $A$ is also closed.

Since $X$ is infinite, there is no finite subcover of the open cover around each singleton point.  Hence, $X$ is not compact.

\subsection{Exercise 21}
Compute
\[ \lim_{x \to \infty} \sin(\sqrt{x+1} - \sqrt{x}). \]

Observe
\begin{align*}
\sqrt{x+1} - \sqrt{x} &= \sqrt{x+1} - \sqrt{x} \cdot \frac{ \sqrt{x+1} + \sqrt{x} }{ \sqrt{x+1} + \sqrt{x}}\\
&= \frac{1}{\sqrt{x+1} + \sqrt{x}}.
\end{align*}

Then
\[ \lim_{x \to \infty} \sin\big(\frac{1}{\sqrt{x+1} + \sqrt{x}}\big) = \sin\big(\lim_{x \to \infty}\frac{1}{\sqrt{x+1} + \sqrt{x}}\big) = 0 \]

\subsection{Exercise 22}
Consider the sequence defined recursively by $a_1 = 1$ and $a_{n+1} = 3 + \frac{a_n}{2}$ for all $n \in \mathbb{N}$.  Prove that $\{a_n\}$ converges.  Find its limit.

Computing a few values of $a_n$ by hand, the sequence might be bound above by $6$.  We can attempt to prove this by induction.

Base case, setting $n=1$
\[ a_1 = 1 < 6 \]

Let $n$ be arbitary and suppose $s_n < 6$.

\begin{align*}
a_{n+1} &= 3 + \frac{a_n}{2} \\
&< 3 + \frac{6}{2} \\
&= 6
\end{align*}

Hence, $a_{n+1}$ holds true and the sequence is bounded.  Furthmore, it is trivially easy to see that this sequence is monotonic.  Hence, the sequence converges.

Since the sequence converges,

\[ \lim_{n \to \infty} 3 + \frac{a_n}{2} = L \]

must satisfy
\[ L = 3 + \frac{L}{2} \]

Hence the limit is $6$.

\subsection{Exercise 23}
Prove that there exists a number $x \in [0, \frac{\pi}{2}]$ such that $2x-1 =
\sin(x^2 + \frac{\pi}{4})$.

Let $f(x) = 2x - 1$, which is continuous function over $[0, \frac{\pi}{2}]$.  Observe \marginnote{The inequality easily follows since the image of $\sin(x)=[-1,1]$}.

\[ f(0) = -1 \leq \sin(x^2 + \frac{\pi}{4}) < \pi - 1 = f(\frac{\pi}{2}) \]

Therefore, by the Intermediate Value Theorem there exists $f(x) = \sin(x^2 + \frac{\pi}{4})$

\end{document}
\grid
