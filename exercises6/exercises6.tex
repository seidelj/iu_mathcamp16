\documentclass{tufte-book}

\usepackage{amsmath, amsthm}
\usepackage{graphicx}
\setkeys{Gin}{width=\linewidth,totalheight=\textheight,keepaspectratio}
\graphicspath{{graphics/}}

\title{Math Camp 2016\\Exercises }
\author{Joe Seidel}
\date{\today}

\usepackage{booktabs}
\usepackage{units}
\usepackage{fancyvrb}
\fvset{fontsize=\normalsize}
\usepackage{multicol}
\usepackage{lipsum}
\usepackage{pdfpages}
\usepackage{tikz}
\usepackage{wasysym}
\usepackage{amssymb}


\newcommand{\doccmd}[1]{\texttt{\textbackslash#1}}% command name -- adds backslash automatically
\newcommand{\docopt}[1]{\ensuremath{\langle}\textrm{\textit{#1}}\ensuremath{\rangle}}% optional command argument
\newcommand{\docarg}[1]{\textrm{\textit{#1}}}% (required) command argument
\newenvironment{docspec}{\begin{quote}\noindent}{\end{quote}}% command specification environment
\newcommand{\docenv}[1]{\textsf{#1}}% environment name
\newcommand{\docpkg}[1]{\texttt{#1}}% package name
\newcommand{\doccls}[1]{\texttt{#1}}% document class name
\newcommand{\docclsopt}[1]{\texttt{#1}}% document class option name
\DeclareMathOperator{\proj}{proj}
\newcommand{\vct}{\mathbf}
\newcommand{\dprod}[2]{\langle \vct{#1}, \vct{#2} \rangle}
\newcommand{\pdv}[2]{\frac{\partial #1}{\partial #2}}
\newtheoremstyle{mytheoremstyle} % name
	{\topsep}		% Space above
	{\topsep}		% Space below
	{\itshape}		% Body font
	{}			% Indent amount
	{\bfseries}	% Theorem head font
	{\textnormal{:}}	% Punctuation after theorem head
	{.5em}		% Space after theorem head
	{}			%Theorem headspec
\theoremstyle{mytheoremstyle}
\newtheorem*{thm}{Thm.}

\newtheoremstyle{mylemstyle} % name
	{\topsep}		% Space above
	{\topsep}		% Space below
	{\itshape}		% Body font
	{}			% Indent amount
	{\bfseries}	% Theorem head font
	{\textnormal{:}}	% Punctuation after theorem head
	{.5em}		% Space after theorem head
	{}			%Theorem headspec
\theoremstyle{mylemstyle}
\newtheorem*{lem}{Lem.}


\newtheoremstyle{mydefstyle} % name
	{\topsep}		% Space above
	{\topsep}		% Space below
	{\normalfont}	% Body font
	{}			% Indent amount
	{\bfseries}	% Theorem head font
	{\textnormal{:}}	% Punctuation after theorem head
	{.5em}		% Space after theorem head
	{}			%Theorem headspec
\theoremstyle{mydefstyle}
\newtheorem*{mydef}{Def.}
\newtheorem*{ex}{E.g.}

\begin{document}

\maketitle
\pagenumbering{gobble}
\newpage
\pagenumbering{arabic}

\subsection{Exercise 1}
Is the set $X=\{ (x,y) \in R^2 \ | \ xy <1 \}$ convex?

Paul Sally Jr's definition of convex sets eliminated some ambiguity over choice of $\lambda$ in the definition provided in the math camp nots.

\mydef[Convex Sets] Let $A$ be a non-empty subset of $\mathbb{R}^n$.  We say $A$ is \textit{convex} if, given any two points $\vct{p},\vct{q} \in A$, the "line segment" with endpoints $\vct{p}$ and $\vct{q}$, that is, the set
\[ \{(1-t)\vct{p} + t\vct{q} \ | \ t\in\mathbb{R}, 0\leq t \leq 1 \} \]
is a subset of $A$.

Hence, choose $\vct{x} = (a, 0)$ and $\vct{y} = (0,a)$, where $a > 0$ and large enough.  Then $\vct{x}, \vct{y} \in X$, but $z = (\lambda a , (1-\lambda)a) \not\in X$ for some $\lambda \in [0,1]$.  Therefore, $X$ is not convex.

\ex $(2,0)$ and $(0,2)$.  The point $(1,1) \in \{z\}$ but $(1,1) \not\in X$.

\subsection{Exercise 3}

Prove Theorem 5.

Let $f$ be concave, $\lambda \in [0,1]$ and suppose $f(x) > f(y)$.

\begin{align}
f(\lambda x + (1-\lambda)y) &\geq \lambda f(x) + (1-\lambda)f(y) \\
&\geq \lambda f(y) + (1-\lambda)f(y)\\
&= f(y) \\
&= \min\{f(x), f(y)\}
\end{align}

Therefore $f$ is quasi-concave.  By the the definition of concave, the inequality becomes strict when $\lambda \in (0,1)$, hence also stricltly quasi-concave.

A similar proof follows for convex implies quasi-convex.  Flip the inequality in $(2)$ and $(3)$ and choose $f(y)>f(x)$, e.g $f(y) = \max\{f(x),f(y)\$.

\end{document}
\grid
