\documentclass{tufte-book}

\usepackage{amsmath, amsthm}
\usepackage{graphicx}
\setkeys{Gin}{width=\linewidth,totalheight=\textheight,keepaspectratio}
\graphicspath{{graphics/}}

\title{Math Camp 2016\\Exercises }
\author{Joe Seidel}
\date{\today}

\usepackage{booktabs}
\usepackage{units}
\usepackage{fancyvrb}
\fvset{fontsize=\normalsize}
\usepackage{multicol}
\usepackage{lipsum}
\usepackage{pdfpages}
\usepackage{tikz}
\usepackage{wasysym}
\usepackage{amssymb}


\newcommand{\doccmd}[1]{\texttt{\textbackslash#1}}% command name -- adds backslash automatically
\newcommand{\docopt}[1]{\ensuremath{\langle}\textrm{\textit{#1}}\ensuremath{\rangle}}% optional command argument
\newcommand{\docarg}[1]{\textrm{\textit{#1}}}% (required) command argument
\newenvironment{docspec}{\begin{quote}\noindent}{\end{quote}}% command specification environment
\newcommand{\docenv}[1]{\textsf{#1}}% environment name
\newcommand{\docpkg}[1]{\texttt{#1}}% package name
\newcommand{\doccls}[1]{\texttt{#1}}% document class name
\newcommand{\docclsopt}[1]{\texttt{#1}}% document class option name
\DeclareMathOperator{\proj}{proj}
\newcommand{\vct}{\mathbf}
\newcommand{\dprod}[2]{\langle \vct{#1}, \vct{#2} \rangle}
\newcommand{\pdv}[2]{\frac{\partial #1}{\partial #2}}
\newtheoremstyle{mytheoremstyle} % name
	{\topsep}		% Space above
	{\topsep}		% Space below
	{\itshape}		% Body font
	{}			% Indent amount
	{\bfseries}	% Theorem head font
	{\textnormal{:}}	% Punctuation after theorem head
	{.5em}		% Space after theorem head
	{}			%Theorem headspec
\theoremstyle{mytheoremstyle}
\newtheorem*{thm}{Thm.}

\newtheoremstyle{mylemstyle} % name
	{\topsep}		% Space above
	{\topsep}		% Space below
	{\itshape}		% Body font
	{}			% Indent amount
	{\bfseries}	% Theorem head font
	{\textnormal{:}}	% Punctuation after theorem head
	{.5em}		% Space after theorem head
	{}			%Theorem headspec
\theoremstyle{mylemstyle}
\newtheorem*{lem}{Lem.}


\newtheoremstyle{mydefstyle} % name
	{\topsep}		% Space above
	{\topsep}		% Space below
	{\normalfont}	% Body font
	{}			% Indent amount
	{\bfseries}	% Theorem head font
	{\textnormal{:}}	% Punctuation after theorem head
	{.5em}		% Space after theorem head
	{}			%Theorem headspec
\theoremstyle{mydefstyle}
\newtheorem*{mydef}{Def.}
\newtheorem*{ex}{E.g.}

\begin{document}

\maketitle
\pagenumbering{gobble}
\newpage
\pagenumbering{arabic}

\subsection{Exercise 3}
Let $f(x,y) = x \ln(x^2 + y^2)$.  Calculate its partial derivatives.

To find $\pdv{f}{y}$, use the chain rule, holding $x$ constant.

\[ \pdv{f}{y} = x \frac{2y}{x^2 + y^2} \]

To find $\pdv{f}{x}$, use the product rule and the chain rule/

\[ \pdv{f}{x} = \ln(x^2+y^2) + \frac{2x^2}{x^2+y^2} \]

\subsection{Exercise 4}
Let $f(x,y,z) = (x^2 + y^2)z^2 + \sin x^2$.  Calculate it's partial derivatives.

To find $\pdv{f}{x}$ use the chain rule for each terms of the sum.

\[ \pdv{f}{x} = 2x(z^2 + \cos x^2) \]

To find $\pdv{f}{y}$ use chain rule on $(x^2+y^2)z^2$ and notice $\sin x^2$ is constant so it becomes $0$.

\[ \pdv{f}{y} = 2yz^2 \]

To find $\pdv{f}{z}$ notice $\sin x^2$ and $(x^2 + y^2)$ are constants hence we
have the form $Bz^2 + C$.

\[ \pdv{f}{z} = 2z(x^2+y^2) \]

\subsection{Exercise 5}
Let $z=z(u,v)= v \ln u$, $u=x^2+y^2$ and $v= \frac{y}{x}$.  Calculate $\pdv{z}{x}$ and $\pdv{z}{y}$.

Observe that $z = z(u,v) = \frac{y}{x} \ln (x^2 + y^2)$ hence

\[ \pdv{z}{x} = -\frac{y \ln(x^2+y^2)}{x^2} + \frac{2y}{x^2+y^2} \]

and

\[ \pdv{z}{y} = \frac{1}{x}(\ln(x^2+y^2) + \frac{2y^2}{x^2+y^2}). \]

\subsection{Exercise 7}
Let $z= e^{xy} \sin(x+y)$.  Calculate $\pdv{z}{x}$ and $\pdv{z}{y}$.

\[ \pdv{z}{x} = e^{xy}(y \sin(x+y) + \cos(x+y) \]
and
\[ \pdv{z}{y} = e^{xy}(x \sin(x+y) + \cos(x+y). \]

\subsection{Exercise 8}
Let $f(x,y) = x^2 + 2xy + y^2$.  Calculate $(\Delta f)(1,2)$.

First find
\[ (\Delta f)(x,y) = (\pdv{f}{x}, \pdv{f}{y}) = (2(x+y), 2(x+y)) \]
then

\[ (\Delta f)(1,2) = (6,6). \]

\subsection{Exercise 9}

Let $F(x) = (x^2+y^3, xy)$.  Calculate the Jacobian $J_F$.

\[ f_1 = x^2 + y^3 \text{ and } f_2=xy \]

\[ J_F =
\begin{bmatrix}
\pdv{f_1}{x} & \pdv{f_1}{y} \\
\pdv{f_2}{x} & \pdv{f_2}{y} \\
\end{bmatrix}
=
\begin{bmatrix}
2x & 3y^2 \\
y & x
\end{bmatrix}
\]

\subsection{Exercise  11}
Prove $(\ln x)'=\frac{1}{x}$ for $x > 0$, provided $(e^x)'=e^x$.

We are given the inverse function of $\ln x$ and it's derivative.  Using the Inverse Function Theorem observe

\[ (\ln x)' = \frac{1}{e^{\ln x}} = \frac{1}{x}. \]
\end{document}
\grid
