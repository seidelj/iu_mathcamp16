\documentclass{tufte-book}

\usepackage{amsmath, amsthm}
\usepackage{graphicx}
\setkeys{Gin}{width=\linewidth,totalheight=\textheight,keepaspectratio}
\graphicspath{{graphics/}}

\title{Math Camp 2016\\Exercises }
\author{Joe Seidel}
\date{\today}

\usepackage{booktabs}
\usepackage{units}
\usepackage{fancyvrb}
\fvset{fontsize=\normalsize}
\usepackage{multicol}
\usepackage{lipsum}
\usepackage{pdfpages}
\usepackage{tikz}
\usepackage{wasysym}
\usepackage{amssymb}

\newcommand{\doccmd}[1]{\texttt{\textbackslash#1}}% command name -- adds backslash automatically
\newcommand{\docopt}[1]{\ensuremath{\langle}\textrm{\textit{#1}}\ensuremath{\rangle}}% optional command argument
\newcommand{\docarg}[1]{\textrm{\textit{#1}}}% (required) command argument
\newenvironment{docspec}{\begin{quote}\noindent}{\end{quote}}% command specification environment
\newcommand{\docenv}[1]{\textsf{#1}}% environment name
\newcommand{\docpkg}[1]{\texttt{#1}}% package name
\newcommand{\doccls}[1]{\texttt{#1}}% document class name
\newcommand{\docclsopt}[1]{\texttt{#1}}% document class option name
\DeclareMathOperator{\proj}{proj}
\newcommand{\vct}{\mathbf}
\newcommand{\dprod}[2]{\langle \vct{#1}, \vct{#2} \rangle}

\newtheoremstyle{mytheoremstyle} % name
	{\topsep}		% Space above
	{\topsep}		% Space below
	{\itshape}		% Body font
	{}			% Indent amount
	{\bfseries}	% Theorem head font
	{\textnormal{:}}	% Punctuation after theorem head
	{.5em}		% Space after theorem head
	{}			%Theorem headspec
\theoremstyle{mytheoremstyle}
\newtheorem*{thm}{Thm.}

\newtheoremstyle{mylemstyle} % name
	{\topsep}		% Space above
	{\topsep}		% Space below
	{\itshape}		% Body font
	{}			% Indent amount
	{\bfseries}	% Theorem head font
	{\textnormal{:}}	% Punctuation after theorem head
	{.5em}		% Space after theorem head
	{}			%Theorem headspec
\theoremstyle{mylemstyle}
\newtheorem*{lem}{Lem.}


\newtheoremstyle{mydefstyle} % name
	{\topsep}		% Space above
	{\topsep}		% Space below
	{\normalfont}	% Body font
	{}			% Indent amount
	{\bfseries}	% Theorem head font
	{\textnormal{:}}	% Punctuation after theorem head
	{.5em}		% Space after theorem head
	{}			%Theorem headspec
\theoremstyle{mydefstyle}
\newtheorem*{mydef}{Def.}
\newtheorem*{ex}{E.g.}

\begin{document}

\maketitle
\pagenumbering{gobble}
\newpage
\pagenumbering{arabic}

\subsection{Exercise 1}
Let
\[ f(x) = \frac{x^2 + x + 1}{e^x} .\]
Calculate $f'(x)$

Observe
\begin{align*}
g(x) &= x^2 + x + 1\\
g'(x) &= 2x + 1\\
h(x) &= e^x\\
h'(x) &= e^x.
\end{align*}

Then rewrite
\[ f(x) = \big(\frac{g}{h}\big)(x). \]

From which it easily follows
\begin{align*}
f'(x) &= \big(\frac{g}{h}\big)'(x)\\
&= \frac{g'(x)h(x) + g(x)h'(x)}{h^2(x)}\\
&= \frac{x-x^2}{e^x}.
\end{align*}

\subsection{Exercise 2}
Let $f(x) = e^{ax} \sin bx$.  Calculate $f'(x)$.

Write
\begin{align*}
g(x) &= e^x\\
h(x) &= ax\\
j(x) &= \sin x \\
k(x) &= bx.
\end{align*}

Then

\[ f(x) = g(h(x))\cdot j(k(x)) = L(x)M(x) = (ML)(x). \]
and
\[f'(x) = L'(x)M(x) + L(x)M'(x) \]

Compute
\begin{align*}
L'(x) &= g'(h(x))h'(x) = e^{ax} \cdot a = ae^{ax}\\
M'(x) &= j'(k(x))k'(x) = \cos bx \cdot d = b \cos bx. \\
\end{align*}

Which leads to the result

\[ f'(x) = e^{ax}(a \sin bx + b \cos bx). \]

\subsection{Exercise 3}
\begin{enumerate}

\item Let
\[ f(x) =
\begin{cases}
x \sin \frac{1}{x} & \ x \neq 0 \\
0 & \ x = 0\\
\end{cases}
\]

Calculate $f'(x)$ for $x \neq 0$ and show that $f(x)$ is not differentiable at $x=0$.

Write
\begin{align*}
g(x) &= x\\
g'(x) &= 1\\
h(x) &= \sin x\\
h'(x) &= \cos x\\
j(x) &= \frac{1}{x}\\
j'(x) &= -\frac{1}{x^2}.
\end{align*}

Observe
\[ f'(x) = g'(x)h(j(x)) = h'(j(x))j'(x)g(x) = \sin \frac{1}{x} - \frac{1}{x} \cos \frac{1}{x}. \]

Finally, since $\frac{1}{x}$ is undefined when $x=0$, $f(x)$ is not differentialable at $0$.

\item Let
\[ f(x) =
\begin{cases}
x^2 \sin \frac{1}{x} & \ x \neq 0 \\
0 & \ x = 0\\
\end{cases}
\]

Write
\begin{align*}
g(x) &= x^2\\
g'(x) &= 2x\\
h(x) &= \sin x\\
h'(x) &= \cos x\\
j(x) &= \frac{1}{x}\\
j'(x) &= -\frac{1}{x^2}.
\end{align*}

Using the same method as in part (i)
\[f'(x) = 2x \sin \frac{1}{x} - \cos\frac{1}{x} . \]

\end{enumerate}

\subsection{Exercise 5}
Prove that
\[ \frac{b-a}{b} < \ln\frac{b}{a} < \frac{b-a}{a} \ (0<a<b).\]

\begin{proof}
First observe $\ln\frac{b}{a} = \ln(b) - \ln(a)$.  Rewriting the inequality

\[ \frac{b-a}{b} < \ln(b) - \ln(a) < \frac{b-a}{a} \]

and diving all the terms by $(b-a)$

\[ \frac{1}{b} < \frac{\ln(b) - \ln(a)}{b-a} < \frac{1}{a}. \]

From the mean value theorem
\[ \frac{\ln(b) - \ln(a)}{b-a} = \ln'(x) = \frac{1}{x} \]

for some $x \in (a,b)$. The inequality is now

\[ \frac{1}{b} < \frac{1}{x} < \frac{1}{a}. \]

Since $b > a \implies \frac{1}{b} < \frac{1}{a}$ the inequality holds.
\end{proof}

\subsection{Exercise 6}
Revisit Chap.1 Exercise 15. Compute

\[ \lim_{x\to 0}\frac{\sqrt{1 + x} - \sqrt{1-x} }{x}  \]

using L'Hospital's rule.

Write
\[ f(x) = \sqrt{1 + x} - \sqrt{1-x} \text{ and } g(x) = x, \]

then
\[ \lim_{x \to 0} f(x) = \lim_{x \to 0} g(x) = 0 \]

making this a problem a prime candidate for L'Hospital's rule, as suggested.

Observe
\[ f'(x) = \frac{1}{2\sqrt{x+1}} - \frac{1}{2\sqrt{1-x}} \]
and
\[ g'(x) = 1. \]

Therefore

\[ \lim_{x\to 0}\frac{\sqrt{1 + x} - \sqrt{1-x} }{x} = \lim_{x\to 0} \frac{f'(x)}{g'(x)} = 0 \]

\subsection{Exercise 7}

Prove that
\[ \lim_{x \to \infty} \Big( 1 + \frac{1}{x} \Big)^x = e.\]

\begin{proof}

First
\[ \log \Big( \big(1 + \frac{1}{x}\big)^x \Big) = x \log(1 + \frac{1}{x}). \]

Write $f(x) = \log(1 + \frac{1}{x})$ and $g(x) = \frac{1}{x}$. Hence

\[ \log\Big(\lim_{x \to \infty} \Big( 1 + \frac{1}{x} \Big)^x \Big) = \lim_{x \to \infty}\frac{f(x)}{g(x)}. \]

Where $\lim_{x \to \infty}f(x) = \lim_{x \to \infty}g(x) = 0$, making this another PRIME candidate for L'Hospital's rule.

Find
\begin{align*}
f'(x) &= \big(\frac{1}{1+\frac{1}{x}}\big)(-\frac{1}{x^2}) \\
g'(x) &= -\frac{1}{x^2}. \\
\end{align*}

Therefore

\[ \lim_{x \to \infty} \frac{f'(x)}{g'(x)} = \lim_{x \to \infty} \frac{1}{1 + \frac{1}{x}} = 1. \]

To recap we have found

\[ \log\Big(\lim_{x \to \infty} \Big( 1 + \frac{1}{x} \Big)^x \Big) = 1 \]

which means

\[ \lim_{x \to \infty} \Big( 1 + \frac{1}{x} \Big)^x = e.\]

\end{proof}

\subsection{Exercise 8}
Let $f(x) = e^x \cos x$.  Prove that $f'' - 2f' + 2f =0$.

First compute
\begin{align*}
f'(x) &= e^x(\cos x - \sin x) \\
f''(x) &= -2e^x \sin x.\\
\end{align*}

Then
\begin{align*}
f'' - 2f' + 2f &= -2e^x \sin x - 2e^x(\cos x - \sin x) + 2e^x \cos x \\
&= 0
\end{align*}

\subsection{Exercise 10}
Find the Taylor polynomial of degree $n$ for $f(x) = \frac{1}{1-x}$, centered at $x=0$.

Observe
\begin{align*}
f'(x) &= \frac{1}{(x-1)^2} = \frac{1!}{(x-1)^2} \\
f''(x) &= \frac{2}{(1-x)^3} = \frac{2!}{(x-1)^3}\\
f'''(x) &= \frac{6}{(1-x)^4}= \frac{3!}{(x-1)^4}\\
f^n(x) &= \frac{n!}{(1-x)^{n+1}}
\end{align*}

Recall
\[ f(\beta) = P(\beta) + \frac{f^{n}(x)}{n!}(\beta - \alpha)^n \]

and let $f(t) = \frac{1}{1-x}$, $\alpha=0$, and $\beta = x$.

Hence

\[ \frac{1}{1-x} = 1 + x + x^2 + x^3 + ... + x^n + \frac{1}{(1-s)^{n+1}}x^{n+1} \]

with $0 < s < x$.

Since
\[ \lim_{x \to 0} \frac{1}{(1-s)^{n+1}}x^{n+1} = 0 \]

this can be rewritten \marginnote{I am not sure about the error term.}

\[ \frac{1}{1-x} = 1 + x + x^2 + x^3 + ... + x^n + o(x^n) \ , \ (x \to 0). \]

\subsection{Exercise 12}

Let $f(x) = \cos^5 \sqrt{1 + x^2}$.  Calculate $f'(x)$.

Write
\begin{align*}
g(x) &= x^5 \\
h(x) &= \cos x \\
j(x) &= \sqrt{x} \\
k(x) &= 1+x^2\\
\end{align*}

then $f(x) = g(h(j(k(x))))$ and
\[ f'(x) = g'(h(j(k(x))))h'(j(k(x)))j'(k(x))k'(x). \]

First
\begin{align*}
g'(x) &= 5x^4 \\
h(x) &= -\sin x \\
j(x) &= \frac{1}{2x} \\
k(x) &= 2x\\
\end{align*}

then put it all together

\[ f'(x) = -\frac{5x \cos^4 \sqrt{1 + x^2} \cdot \sin \sqrt{1 + x^2}}{\sqrt{1+x^2}} \]

\subsection{Exercise 13}

Use the mean value theorem to prove that

\[ |\sin x - \sin y| \leq |x - y| \]

\begin{proof}

Let $f(x) = \sin x$ which is continous over some interval $[x,y]$ and differentiable over $(x,y)$, then by the mean value theorem

\[ f'(x) = \frac{f(x) - f(y)}{x-y} \]

for some $x \in (x,y)$.  Taking the absolute value of both sides and noting that $|f'(x) \leq 1\|$

\[ |f(x) - f(y)| = |x-y||f'(x)| \leq |x-y| \]

\end{proof}

\subsection{Exercise 14}
Compute
\[ \lim_{x \to 1} \big( \frac{1}{\ln x} - \frac{1}{x-1} \big). \]

Rewrite

\[ \frac{1}{\ln x} - \frac{1}{x-1} = \frac{x-1-\ln x}{\ln x (x-1)} \]

where $g(x) = x - 1 - \ln x$ and $h(x) = \ln x(x-1)$.

Then
\[ \lim_{x \to 1} f(x) = \lim_{x \to 1} \frac{g(x)}{h(x)} =\frac{0}{0}. \]

Observe
\begin{align*}
g'(x) &= 1- \frac{1}{x}\\
h'(x) &= \frac{1}{x}(x-1) + \ln x\\
\end{align*}

also tend to $0$ as $x \to 1$.  So we can take the second derivative

\begin{align*}
g''(x) &= \frac{1}{x^2} \\
h''(x) &= \frac{x-1}{x^2} + \frac{2}{x}.\\
\end{align*}

Then observe, by L'Hospital's rule

\[ \lim_{x \to 1} f(x) = \lim_{x \to 1} \frac{g'(x)}{h'(x)} = \lim_{x \to 1} \frac{g''(x)}{h''(x)} = \frac{1}{2} \]

\subsection{Exercise 15}
Find the Taylor polynomial of degree three for $f(x) = \sin x$, centered at $x=\frac{5\pi}{6}$.

Let $f(t) = \sin t$, $\alpha = \frac{5\pi}{6}$, and $\beta=x$.

\begin{align*}
f'(t) &= \cos t\\
f''(t) &= -\sin t\\
f^{(3)}(t) &= -\cos t\\
\end{align*}

Writing out the polynomial
\begin{equation*}
\sin x = \frac{\sin \alpha}{0!}(x-\frac{5\pi}{6})^0 + \frac{\cos \alpha}{1!}(x-\frac{5\pi}{6})^1 + \frac{-\sin \alpha}{2!}(x-\frac{5\pi}{6})^2 + \frac{-\cos \alpha}{3!}(x-\frac{5\pi}{6})^3
\end{equation*}

Inserting the values as defined,
\begin{equation*}
\sin x = \frac{1}{2} - \frac{\sqrt{3}}{2}(x-\frac{5\pi}{6}) - \frac{(x-\frac{5\pi}{6})^2}{4} + \frac{\sqrt{3}}{12}(x-\frac{5\pi}{6})^3.
\end{equation*}
\end{document}
\grid
