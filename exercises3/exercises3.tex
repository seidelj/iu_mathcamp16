\documentclass{tufte-book}

\usepackage{amsmath, amsthm}
\usepackage{graphicx}
\setkeys{Gin}{width=\linewidth,totalheight=\textheight,keepaspectratio}
\graphicspath{{graphics/}}

\title{Math Camp 2016\\Exercises }
\author{Joe Seidel}
\date{\today}

\usepackage{booktabs}
\usepackage{units}
\usepackage{fancyvrb}
\fvset{fontsize=\normalsize}
\usepackage{multicol}
\usepackage{lipsum}
\usepackage{pdfpages}
\usepackage{tikz}
\usepackage{wasysym}
\usepackage{amssymb}

\newcommand{\doccmd}[1]{\texttt{\textbackslash#1}}% command name -- adds backslash automatically
\newcommand{\docopt}[1]{\ensuremath{\langle}\textrm{\textit{#1}}\ensuremath{\rangle}}% optional command argument
\newcommand{\docarg}[1]{\textrm{\textit{#1}}}% (required) command argument
\newenvironment{docspec}{\begin{quote}\noindent}{\end{quote}}% command specification environment
\newcommand{\docenv}[1]{\textsf{#1}}% environment name
\newcommand{\docpkg}[1]{\texttt{#1}}% package name
\newcommand{\doccls}[1]{\texttt{#1}}% document class name
\newcommand{\docclsopt}[1]{\texttt{#1}}% document class option name
\DeclareMathOperator{\proj}{proj}
\newcommand{\vct}{\mathbf}
\newcommand{\dprod}[2]{\langle \vct{#1}, \vct{#2} \rangle}

\newtheoremstyle{mytheoremstyle} % name
	{\topsep}		% Space above
	{\topsep}		% Space below
	{\itshape}		% Body font
	{}			% Indent amount
	{\bfseries}	% Theorem head font
	{\textnormal{:}}	% Punctuation after theorem head
	{.5em}		% Space after theorem head
	{}			%Theorem headspec
\theoremstyle{mytheoremstyle}
\newtheorem*{thm}{Thm.}

\newtheoremstyle{mylemstyle} % name
	{\topsep}		% Space above
	{\topsep}		% Space below
	{\itshape}		% Body font
	{}			% Indent amount
	{\bfseries}	% Theorem head font
	{\textnormal{:}}	% Punctuation after theorem head
	{.5em}		% Space after theorem head
	{}			%Theorem headspec
\theoremstyle{mylemstyle}
\newtheorem*{lem}{Lem.}


\newtheoremstyle{mydefstyle} % name
	{\topsep}		% Space above
	{\topsep}		% Space below
	{\normalfont}	% Body font
	{}			% Indent amount
	{\bfseries}	% Theorem head font
	{\textnormal{:}}	% Punctuation after theorem head
	{.5em}		% Space after theorem head
	{}			%Theorem headspec
\theoremstyle{mydefstyle}
\newtheorem*{mydef}{Def.}
\newtheorem*{ex}{E.g.}

\begin{document}

\maketitle
\pagenumbering{gobble}
\newpage
\pagenumbering{arabic}

\subsection{Exercise 4}
Compute
\[ \int_0^1 (e^x + x) dx. \]

First find $F(x)$ the anti derivative of $(e^x +x)$.

\[ F(t) = e^t - \frac{t^2}{2}. \]

Apply the fundemental theorem of calculus

\[ \int_0^1 (e^x + x) dx = F(1) - F(0) = e - \frac{1}{2}. \]

\subsection{Exercise 5}
Revisit chapter 2, exercise 3(ii).  Let

\[ f(x) =
\begin{cases}
-\cos \frac{1}{x} + 2 \sin \frac{1}{x}, & \ 0 < x \leq 1\\
0, & \ x=0\\
\end{cases}
\]

Compute $\int_0^1 f(x) dx$.

First, find the $F(x)$. Write
\[ \int f(x) dx = \int 2x \sin\frac{1}{x} + \int -\cos \frac{1}{x} dx. \]

Integrate $\int (-\cos \frac{1}{x}) dx$ by parts\marginnote{$\int fdg = fg - \int gdf$} where

\begin{align*}
f &= x^2\\
dg &= -\frac{\cos \frac{1}{x}}{x^2}dx \\
g &= \sin{\frac{1}{x}}\\
df &= 2xdx.\\
\end{align*}

Now
\[ \int (-\cos \frac{1}{x}) = x^2\sin \frac{1}{x} - \int 2x \sin\frac{1}{x} dx \]
therefore

\[ F(x) =
\begin{cases}
x^2 \sin \frac{1}{x} & 0 < x \leq 1\\
0 & x=0\\
\end{cases}
\]

Applying the fundemental theorem of calculus.

\[ \int_0^1 f(x) dx = F(1) - F(0) = \sin 1 \]


\end{document}
\grid
